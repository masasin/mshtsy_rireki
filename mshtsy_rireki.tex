\documentclass[afour,stamp,picture]{rireki}

\begin{document}
\date{\today}

\surname{勝~者}{まさひと}
\forename{神~優}{しんゆう}
\surnameForeign{Nassar}{なっさー}
\forenameForeign{Jean}{じょん}

\birthday{1990}{11}{5}
\gender{男}

\postalcode{615-8177}
\address{京都府京都市西京区樫原下ノ町23−1此君園荘B-2号室}{きょうとふきょうとしにしきょうくかたぎはらしものちょう}
\phone{080-8333-6065}
\email{mshtsy@gmail.com}
\otheraddress{同~上}{}


% Background
\background{2005}{ 9}{サジェッス高等学校 入学}
\background{2008}{ 7}{サジェッス高等学校 (国際バカロレア資格) 卒業}
\background{2008}{ 9}{ウォータールー大学工学部メカトロニクス学科 入学}
\background{2013}{ 6}{ウォータールー大学工学部メカトロニクス学科 卒業}
\background{2013}{ 7}{京都大学大学院工学研究科機械理工学専攻 (研究生) 入学}
\background{2014}{ 3}{京都大学大学院工学研究科機械理工学専攻 (研究生) 卒業}
\background{2014}{ 4}{京都大学大学院工学研究科機械理工学専攻 (工学修士) 入学}
\background{2017}{ 3}{京都大学大学院工学研究科機械理工学専攻 (工学修士) 修了}
\BackgroundLastNewLine

% Career
\career{2009}{ 1}{シエラ建設会社 入社 (有償インターン・四ヶ月)} 
\career{2009}{ 9}{移動ロボット研究室 配属 (有償インターン・四ヶ月)} 
\career{2010}{ 5}{マルチスケール積層造形研究室 配属 (有償インターン・四ヶ月):細胞等} 
\career{2011}{ 1}{IMS株式会社 入社 (有償インターン・四ヶ月):電子基板} 
\career{2011}{ 9}{KQS有限会社 入社 (有償インターン・四ヶ月):自転車設計} 
\career{2012}{ 5}{Starquip有限会社 入社 (有償インターン・四ヶ月):工業空気圧機器設計} 

% Certs
\license{2010}{12}{普通自動車 第一種免許 取得 (レバノン)}
\license{2012}{1}{応急処置 心肺蘇生(CPR-HCP)資格 取得}
\license{2013}{6}{技術士研修(Engineer in Training)資格 取得}
\license{2017}{3}{日本機械学会 正員 入会}
\license{2017}{7}{普通自動車 仮運転免許 取得 (カナダアルバータ州)}

% PR
\subjects{プログラミング,ロボット開発,コンピューター力,言語力.研究テーマ:「過去画像を用いた飛行ロボットの遠隔操作手法」.}
\selfintro{新しい事を速く習得し,適応能力が高いと思っております.仕事や勉強は一生懸命取り組んで,結果を重視します.}
\hobbies{宇宙,飛行,自転車,登山,言語,プログラミング,読書,料理.}
\strengths{プログラミング (Python [Scipyスタックを含む],ROS, C++, \LaTeX 等),3Dソフト (Solidworks, Inventor, AutoCAD等),OS (Linux, Windows),ロボット開発,言語力 (11カ国語).}
%\motive{宇宙の関心でリモートセンシングやGISの興味が生まれて,未だ経験はないのに活かしたいなと思います.しかも,ドローンを使って危険な仕事がなくなるように頑張りたいです.}
\desire{給料:ご相談させて頂きたいと思っております\\
        職種:プロセス開発・自動化,機械学習・人工知能,プログラミング,ロボット開発}

% Commute
\commute{約 1 時間}
\dependents{0人}
\spouse{なし}
\spousedependence{なし}

% Guardian
%\guardian[やまだ かずお]{山田 和夫}
%\guardianpostalcode{123-4567}
%\guardianaddress[ほごしゃのじゅうしょ]{保護者の住所}
%\guardianphone{0123-45-6789}

% Recycled paper
% \recycled

\end{document}
